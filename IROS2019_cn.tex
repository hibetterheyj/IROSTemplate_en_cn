%%%%%%%%%%%%%%%%%%%%%%%%%%%%%%%%%%%%%%%%%%%%%%%%%%%%%%%%%%%%%%%%%%%%%%%%%%%%%%%%
%2345678901234567890123456789012345678901234567890123456789012345678901234567890
%        1         2         3         4         5         6         7         8

\documentclass[letterpaper, 10 pt, conference]{ieeeconf}  % 如需a4paper注释本行

%\documentclass[a4paper, 10pt, conference]{ieeeconf}      % 如需a4paper使用本行

\IEEEoverridecommandlockouts                              % 本命令用于
                                                          % 如果要使用\thanks

\overrideIEEEmargins                                      % N需要满足打印机要求

%如果您遇到以下错误:
%错误1010 PDF文件可能已损坏(无法打开PDF文件)或
%错误1000解析内容流时发生错误。 无法分析PDF文件。
%这是pdfLaTeX转换过滤器的已知问题。 无法使用acrobat reader打开该文件。
%请使用下面的替代方法之一,通过取消注释其中一个来避免此错误。
%\pdfobjcompresslevel=0
%\pdfminorversion=4

% 请参阅文件中稍后的\addtolength命令以平衡栏长度
% 在文档的最后一页上

% The following packages can be found on http:\\www.ctan.org
%\usepackage{graphics} % for pdf, bitmapped graphics files
%\usepackage{epsfig} % for postscript graphics files
%\usepackage{mathptmx} % assumes new font selection scheme installed
%\usepackage{times} % assumes new font selection scheme installed
%\usepackage{amsmath} % assumes amsmath package installed
%\usepackage{amssymb}  % assumes amsmath package installed
\usepackage{xeCJK}

\title{\LARGE \bf
IEEE 论文准备*
}


\author{Albert Author$^{1}$ and Bernard D. Researcher$^{2}$% <-this % stops a space
\thanks{*This work was not supported by any organization}% <-this % stops a space
\thanks{$^{1}$Albert Author is with Faculty of Electrical Engineering, Mathematics and Computer Science,
        University of Twente, 7500 AE Enschede, The Netherlands
        {\tt\small albert.author@papercept.net}}%
\thanks{$^{2}$Bernard D. Researcheris with the Department of Electrical Engineering, Wright State University,
        Dayton, OH 45435, USA
        {\tt\small b.d.researcher@ieee.org}}%
}


\begin{document}



\maketitle
\thispagestyle{empty}
\pagestyle{empty}


%%%%%%%%%%%%%%%%%%%%%%%%%%%%%%%%%%%%%%%%%%%%%%%%%%%%%%%%%%%%%%%%%%%%%%%%%%%%%%%%
\begin{abstract}

该电子文档是一份模板。纸张的各个组成部分[标题,文字,标题等]已经在样式表中定义,如本文档中给出的部分所示。

\end{abstract}


%%%%%%%%%%%%%%%%%%%%%%%%%%%%%%%%%%%%%%%%%%%%%%%%%%%%%%%%%%%%%%%%%%%%%%%%%%%%%%%%
\section{介绍}

该模板为作者提供了准备论文电子版所需的大部分格式规范。所有标准论文组件的指定原因有三个:(1)格式化论文的易用性,(2)自动编译出符合当前或以后要求的电子文档,以及(3)会议论文的整体一致性。边距,列宽,行间距和类型样式都是内置的;本文档中提供了类型样式的示例,并在示例后的括号内以斜体字标识。虽然提供了各种表格文本样式,但是没有规定一些组件,例如多级方程式,图形和表格。格式化程序需要创建这些组件,并遵循以下适用的标准。

\section{论文提交步骤}

\subsection{选择模板 (Heading 2)}

首先,确认您的论文纸张尺寸正确。此模板已针对美国信纸尺寸的输出进行了定制。如果适当修改纸张尺寸设置,它可用于A4纸张尺寸。

\subsection{始终遵守规范}

该模板用于格式化纸张并设置文本样式。规定了所有边距,列宽,行间距和文本字体; 请不要改变它们。你可能会注意到特殊性。例如,此模板中的顶部边距按比例测量比通常情况更多。此边距和其他边距是人为设定的,使用的规范可以将您的论文作为整个过程的一部分,而不是作为独立文档。请不要修改任何当前的名称。

\section{数学}

在开始格式化纸张之前,请先将内容写入并另存为单独的文本文件。保持文本和图形文件分开,直到文本格式化和样式化。不要使用硬标签,并将硬回车的使用限制为段落末尾的一个返回。不要在论文的任何地方添加任何类型的分页。 不要对文本头进行编号 - 模板将为您执行此操作。

最后,在格式化之前完成内容和组织编辑。校对拼写和语法时请注意以下事项:

\subsection{缩略词} 

即使已在摘要中定义,在文本中第一次使用缩略词时,也应该定义缩略词。不必定义诸如IEEE,SI,MKS,CGS,sc,dc和rms的缩写。除非不可避免,否则不要在标题或标题中使用缩写。

\subsection{单位}

\begin{itemize}

\item 使用SI(MKS)或CGS作为主要单位。(鼓励SI单位。)英语单位可用作辅助单位(括号内)。例外情况是使用英制单位作为交易中的标识符,例如3.5英寸磁盘驱动器。
\item 避免组合SI和CGS单位,例如奥斯特的电流和奥斯特的磁场。 这通常会导致混淆,因为方程在尺寸上不平衡。 如果必须使用混合单位,请清楚地说明在等式中使用的每个数量的单位。
\item 不要混合完整的拼写和单位缩写:Wb / m2或每平方米的网络,而不是webers / m2。 在单词出现在文本中时拼出单位:……几亨利,不是……几个H.
\item 在小数点前使用零:0.25,而不是.25。 使用cm3,而不是cc。(bullet list)

\end{itemize}


\subsection{方程}

方程式是该模板的规定规范的例外。 您需要确定是否应使用Times New Roman或Symbol字体键入等式(请不要使用其他字体)。要创建多层方程,可能需要将方程式视为图形,并在纸张样式后将其插入到文本中。数字方程连续。括号内的公式编号是使用右侧制表位将右侧位置齐平,如(1)所示。为了使方程更紧凑,可以使用 solidus ( / ),exp函数或适当的指数。斜体表示数量和变量的罗马符号,但不是希腊符号。使用长划线而不是连字符表示减号。用逗号或句点表示句点,当它们是句子的一部分时,如

$$
\alpha + \beta = \chi \eqno{(1)}
$$

请注意,使用中心制表位使等式居中。确保等式中的符号已在等式之前或之后定义。 使用(1) ,而不是Eq.(1)或等式(1),除了在句子的开头:等式(1) 是 ……

\subsection{一些常见错误}
\begin{itemize}


\item 单词数据是复数,而不是单数。
\item 真空渗透率的下标0和其他常见的科学常数,下标格式为零,而不是小写字母o。
\item 在美式英语中,逗号,半/冒号,句号,问号和感叹号只有在引用完的思想或名称时才会出现在引号内,例如标题或完整引用。当使用引号而不是粗体或斜体字体来突出显示单词或短语时,标点符号应出现在引号之外。句子末尾的括号短语或语句在右括号之外标点(如下所示)。 (在括号内加上括号内的句子。)
\item 图中的图是插图(inset),而不是插入(insert)。替代这个词比交替使用这个词更优先(除非你的意思是交替出现)。
\item 不要使用这个词本质上是近似或有效的。
\item 在你的论文题目中,如果使用的单词可以准确地替换使用的单词,则将u大写; 如果没有,继续使用小写。
\item 要注意同音异义词的不同含义影响和效果,补充和赞美,谨慎和离散,主要和原则。
\item 不要混淆暗示和推断。
\item 前缀non不是单词; 它应该加入它修改的单词,通常没有连字符。
\item 在拉丁语缩写 et al. 的 et 之后没有句号。
\item 缩略语 i.e. 意思为“也即”,缩略语 e.g. 意思为“举例”。

\end{itemize}


\section{使用模板}

使用此示例文档作为LaTeX源文件创建文档。将此文件保存为{\bf root.tex}。您必须确保使用此发行版附带的 cls文件。如果使用不同的样式文件,则无法获得所需的边距。另请注意,在创建输出PDF文件时,源文件只是其中的一部分。{\it 你的 \TeX\ $\rightarrow$ PDF编译器决定输出文件尺寸。即使您使用所有的规范在源代码中输出一个信件大小的文件。如果您的过滤器设置为生成A4,那么您只能得到A4输出。}

在使用\TeX,无法考虑所有可能的情况。如果您使用的是多个\TeX\文件,则必须确保"MAIN"源文件被称为root.TeX。如果您的会议使用的是 PaperPlaza 内置的\TeX\到PDF转换工具,这一点尤为重要。

\subsection{标题等}

文本标题以关系层次为基础组织主题。例如,论文标题是正文文本标题,因为所有后续材料都涉及并阐述了这一主题。如果有两个或更多子主题,则应使用下一级标题题头(大写罗马数字),相反,如果不存在至少两个子主题,则不应引入子标题。规定了标题1、标题2、标题3和标题4的样式。

\subsection{图片表格}

定位图片和表格:将图片和表格放在栏的顶部和底部。避免将它们放在栏的中间。大型图片和表格可以跨越两栏。图片标题应低于图片;表格标题应放在表格上方。在正文引用之后插入图片和表格。即使在句子的开头,也要使用图1(Fig. 1)的缩写。

\begin{table}[h]
\caption{表格范例}
\label{table_example}
\begin{center}
\begin{tabular}{|c||c|}
\hline
One & Two\\
\hline
Three & Four\\
\hline
\end{tabular}
\end{center}
\end{table}


   \begin{figure}[thpb]
      \centering
      \framebox{\parbox{3in}{我们建议您使用文本框插入图形(理想情况下是300 dpi TIFF或EPS文件,嵌入所有字体),因为在文档中,此方法比直接插入图片更稳定。
}}
      %\includegraphics[scale=1.0]{figurefile}
      \caption{Inductance of oscillation winding on amorphous
       magnetic core versus DC bias magnetic field}
      \label{figurelabel}
   \end{figure}
   

图片标签(Labels):使用8号Times New Roman作为标签。在编写图片坐标轴标签时使用单词而不是符号或缩写,以避免混淆读者。例如,写出数量Magnetization,或Magnetization, M,而不仅仅是M.如果在标签中包含单位,则将它们显示在括号内。不要仅用单位标记坐标轴。在该示例中,应写Magnetization (A/m)或Magnetization {A[m(1)]},而不仅仅是A/m。不要使用数量和单位的比率标记坐标轴。例如,应写Temperature (K),而不是Temperature/K。

\section{结论}

结论部分不是必需的。虽然结论可能会回顾论文的要点,但不要将摘要复制为结论。 结论可能会详细说明工作的重要性或建议应用和扩展。

\addtolength{\textheight}{-12cm}   
% 此命令用于手动平衡文档最后一页上的列长度。它会将最后一页的文本高度缩短适当的数量。此命令直到下一页才会生效,因此它应该出现在最后一页之前的页面上。 确保不要过多地缩短文本高度。

%%%%%%%%%%%%%%%%%%%%%%%%%%%%%%%%%%%%%%%%%%%%%%%%%%%%%%%%%%%%%%%%%%%%%%%%%%%%%%%%



%%%%%%%%%%%%%%%%%%%%%%%%%%%%%%%%%%%%%%%%%%%%%%%%%%%%%%%%%%%%%%%%%%%%%%%%%%%%%%%%



%%%%%%%%%%%%%%%%%%%%%%%%%%%%%%%%%%%%%%%%%%%%%%%%%%%%%%%%%%%%%%%%%%%%%%%%%%%%%%%%
\section*{APPENDIX}

附录应出现在致谢之前。

\section*{致谢}

在美国,acknowledgment单词的推荐拼写g后是没有e的。请避免笨拙的表达,如我们其中一人(R. B. G.)感谢…… 相反,请尝试使用R. B. G.感谢。将支持机构致谢放在第一页的无编号脚注。



%%%%%%%%%%%%%%%%%%%%%%%%%%%%%%%%%%%%%%%%%%%%%%%%%%%%%%%%%%%%%%%%%%%%%%%%%%%%%%%%

参考文献对读者很重要;因此每个引用必须完整和正确。如果可能的话,参考文献应是常用的出版物。



\begin{thebibliography}{99}

\bibitem{c1} G. O. Young, Synthetic structure of industrial plastics (Book style with paper title and editor),    in Plastics, 2nd ed. vol. 3, J. Peters, Ed.  New York: McGraw-Hill, 1964, pp. 1564.
\bibitem{c2} W.-K. Chen, Linear Networks and Systems (Book style).      Belmont, CA: Wadsworth, 1993, pp. 123135.
\bibitem{c3} H. Poor, An Introduction to Signal Detection and Estimation.   New York: Springer-Verlag, 1985, ch. 4.
\bibitem{c4} B. Smith, An approach to graphs of linear forms (Unpublished work style), unpublished.
\bibitem{c5} E. H. Miller, A note on reflector arrays (Periodical styleAccepted for publication), IEEE Trans. Antennas Propagat., to be publised.
\bibitem{c6} J. Wang, Fundamentals of erbium-doped fiber amplifiers arrays (Periodical styleSubmitted for publication), IEEE J. Quantum Electron., submitted for publication.
\bibitem{c7} C. J. Kaufman, Rocky Mountain Research Lab., Boulder, CO, private communication, May 1995.
\bibitem{c8} Y. Yorozu, M. Hirano, K. Oka, and Y. Tagawa, Electron spectroscopy studies on magneto-optical media and plastic substrate interfaces(Translation Journals style), IEEE Transl. J. Magn.Jpn., vol. 2, Aug. 1987, pp. 740741 [Dig. 9th Annu. Conf. Magnetics Japan, 1982, p. 301].
\bibitem{c9} M. Young, The Techincal Writers Handbook.  Mill Valley, CA: University Science, 1989.
\bibitem{c10} J. U. Duncombe, Infrared navigationPart I: An assessment of feasibility (Periodical style), IEEE Trans. Electron Devices, vol. ED-11, pp. 3439, Jan. 1959.
\bibitem{c11} S. Chen, B. Mulgrew, and P. M. Grant, A clustering technique for digital communications channel equalization using radial basis function networks, IEEE Trans. Neural Networks, vol. 4, pp. 570578, July 1993.
\bibitem{c12} R. W. Lucky, Automatic equalization for digital communication, Bell Syst. Tech. J., vol. 44, no. 4, pp. 547588, Apr. 1965.
\bibitem{c13} S. P. Bingulac, On the compatibility of adaptive controllers (Published Conference Proceedings style), in Proc. 4th Annu. Allerton Conf. Circuits and Systems Theory, New York, 1994, pp. 816.
\bibitem{c14} G. R. Faulhaber, Design of service systems with priority reservation, in Conf. Rec. 1995 IEEE Int. Conf. Communications, pp. 38.
\bibitem{c15} W. D. Doyle, Magnetization reversal in films with biaxial anisotropy, in 1987 Proc. INTERMAG Conf., pp. 2.2-12.2-6.
\bibitem{c16} G. W. Juette and L. E. Zeffanella, Radio noise currents n short sections on bundle conductors (Presented Conference Paper style), presented at the IEEE Summer power Meeting, Dallas, TX, June 2227, 1990, Paper 90 SM 690-0 PWRS.
\bibitem{c17} J. G. Kreifeldt, An analysis of surface-detected EMG as an amplitude-modulated noise, presented at the 1989 Int. Conf. Medicine and Biological Engineering, Chicago, IL.
\bibitem{c18} J. Williams, Narrow-band analyzer (Thesis or Dissertation style), Ph.D. dissertation, Dept. Elect. Eng., Harvard Univ., Cambridge, MA, 1993. 
\bibitem{c19} N. Kawasaki, Parametric study of thermal and chemical nonequilibrium nozzle flow, M.S. thesis, Dept. Electron. Eng., Osaka Univ., Osaka, Japan, 1993.
\bibitem{c20} J. P. Wilkinson, Nonlinear resonant circuit devices (Patent style), U.S. Patent 3 624 12, July 16, 1990. 






\end{thebibliography}




\end{document}
